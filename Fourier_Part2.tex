\documentclass[10pt]{beamer}
\usefonttheme{professionalfonts}
\usepackage[utf8]{inputenc}
\usepackage[T1]{fontenc}
\usepackage{lmodern}
\usepackage[english]{babel}
\usepackage{amsmath}
\usepackage{amsfonts}
\usepackage{upgreek}
\usepackage{amssymb}
\usepackage{graphicx}
\usepackage{mathtools}
\usetheme{AnnArbor}
%\usetheme{UniKlu}

\newcommand{\iu}{{i\mkern1mu}}
\newcommand{\ju}{{j\mkern1mu}}

\begin{document}



\author{\textcolor{orange}{Balaji Gopalakrishnan}}
\title{Fourier Transforms - Part II}
\subtitle{\huge Review of math and basic Fourier Series}
%\logo{"index.png"}

\institute{CIT Alumni Group}
\date{May 15, 2020}
%\subject{Hi}
%\setbeamercovered{transparent}
%\setbeamertemplate{navigation symbols}{}
%\titlegraphic{\includegraphics[width=2cm]{index.jpeg}}


\begin{frame}[plain]
	\maketitle
	%\date{05/19/2020}
\end{frame}

\begin{frame}
	\frametitle{Topics in Part II \hspace{25pt}} 
	We will cover the following topics in the next 30 minutes. We will only briefly discuss each of the areas to serve as a brush up of the concepts. This is not a rigorous math class. \\
	
\begin{list}{$\cap$}{}
	\item Quick review of Part 1 of the series
	\item Quick review of Vector Algebra
	\item Fourier Series
	\item Fourier Transform - General Formula and Intutive Explanation
	\item Fourier Transform Visual examples
	\item Digital Fourier Transform and Fast Fourier Transform
\end{list}
	
\end{frame}

\begin{frame}
	\frametitle{ Vector Mathematics Intro}
	
	\begin{list}{$\cap$}{}
		\item The important characteristic of a vector quantity is that it has both a magnitude (or size) and a direction.Both of these properties must be given in order to specify a vector completely.
		\item An example of a vector quantity is velocity.  It tells us how fast is the object traveling and the direction of travel
		\item A quantity with magnitude alone, but no direction, is not a vector. It is called a scalar.One example of a scalar is distance. This tells us how far we are from a fixed point, but does not give us any information about the direction.
		\item A vector quantity is denoted as  $\vec{A}$
		
		\item A vector quantity can be represented in $\{x,y\} $ format or $R\angle$ format
	\end{list}
\end{frame}

\begin{frame}
	\frametitle{ Vector Operations}
	
	We will see visual examples of this.
	
	\begin{list}{$\cap$}{}
		\item Addition 
		\item Subtraction
		\item Multiplication
		
		\begin{enumerate}
			\item Dot Product - Denoted as: $A \odot B$
			\item Cross Product - Denoted as $A \otimes B$
		\end{enumerate}
	
	\end{list}

\end{frame}

\begin{frame}
	\frametitle{Visual Examples of Algebraic operations}

	We will quickly see in the graphs via Geogebra, how vector addition and subtraction work. \\
	
	We will briefly cover dot product and cross product, but for this discussion we do not need to worry about products.\\
	
	
\end{frame}

\begin{frame}
	\frametitle{Geometric Series Proof}
	Geometric series GS = $a,ar^1 ,ar^2, ar^3 ...... ar^{n-1}$
	The sum of this series is 
	$$S_n = a+ar^1 +ar^2+ar^3 ...... + ar^{n-1} $$
	Multiply both sides by r
	$$rS_n = ar^1+ar^2 +ar^3+ar^4 ...... + ar^{n} $$
	Subtract both equations
	$$S_n(1-r) = a(1-r^n)$$
	$$Sn = \frac{a(1-r^n)}{(1-r)}$$
	
	
\end{frame}



\begin{frame}
	\frametitle{Fourier Series Expansions - DISCRETE TIME FOURIER SERIES}
	
		\begin{list}{$\cap$}{}
			
			\item Lets assume $x[k]$ is periodic with period $N_0$
			\item Goal is to write $x[k]$ as a series of complex sinusoids such as $$x[k] = \sum_{0}^{r}{D_r e^ {\ju r\omega_0 k}}$$
			\item Where $\omega_0 = 2\pi/N_0$ which is also called as fundamental frequency of $x[k]$
			
			\item Also recall $$ e^ {\ju {r +N_0}\omega_0 k} =  e^ {\ju {r }\omega_0 k} * e^ {\ju {N_0}\omega_0 k} = e^ {\ju {r }\omega_0 k} $$

		\end{list}
		
\end{frame}

\begin{frame}
	\frametitle{Continued}
$$x[k] = \sum_{0}^{N_0 -1}{D_r e^ {\ju r\omega_0 k}}$$
Multiply both sides by the sum of all frequencies : $\sum_{r=0}^{N_0-1}e^{-\ju m \omega k}$

$$\sum_{k=0}^{N_0-1}{e^{-\ju m \omega k} x[k]} = {\sum_{k=0}^{N_0-1} {\sum_{r=0}^{N_0 -1}}{D_r e^ {\ju r\omega_0 k} {e^ {-\ju m\omega_0 k}}}} =  {\sum_{r=0}^{N_0-1}}{ {D_r}\sum_{k=0}^{N_0 -1} { e^ {\ju {(r-m)}\omega_0 k}}} $$ 

When $r \ne m $ the second part of the series collapses to zero because of geometric series
$$\sum_{k=0}^{N_0 -1} { e^ {\ju {(r-m)}\omega_0 k}} = {\frac{e^{\ju(r-m)\omega_0N_0}-1 }{e^{\ju(r-m)\omega_0}-1 } }  $$  
as $\omega_0 = \frac{2*\pi}{N_0}$ , the above equation becomes 0 and when  $m = r$ , the equation becomes

$$D_m = \frac{1}{N_0} \sum_{k=0}^{N_0 -1} x[k] {e^{\ju m\omega_0k}}$$


	

\end{frame}

\begin{frame}
	\frametitle{Fourier Series Expansions - DISCRETE TIME FOURIER SERIES -2 }
	
	
	
\end{frame}


\begin{frame}
	\frametitle{Fourier Transform}
	
	Fouries Transform is just a process of taking a function and making into a fourier series.  
\end{frame}

\end{document}
