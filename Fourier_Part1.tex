\documentclass[10pt]{beamer}
\usefonttheme{professionalfonts}
\usepackage[utf8]{inputenc}
\usepackage[T1]{fontenc}
\usepackage{lmodern}
\usepackage[english]{babel}
\usepackage{amsmath}
\usepackage{amsfonts}
\usepackage{upgreek}
\usepackage{amssymb}
\usepackage{graphicx}
\usepackage{mathtools}
\usetheme{AnnArbor}
%\usetheme{UniKlu}

\newcommand{\iu}{{i\mkern1mu}}

\begin{document}



\author{\textcolor{orange}{Balaji Gopalakrishnan}}
\title{Fourier Transforms - Part I }
\subtitle{\huge Review of Trigonometry \& Calculus}
%\logo{"index.png"}

\institute{CIT Alumni Group}
\date{May 15, 2020}
%\subject{Hi}
%\setbeamercovered{transparent}
%\setbeamertemplate{navigation symbols}{}
%\titlegraphic{\includegraphics[width=2cm]{index.jpeg}}


\begin{frame}[plain]
	\maketitle
	%\date{05/19/2020}
\end{frame}

\begin{frame}
	\frametitle{Topics in Part I \hspace{25pt}} 
	We will cover the following topics in the next 30 minutes. We will only briefly discuss each of the areas to serve as a brush up of the concepts. This is not a rigorous math class. \\
	
\begin{list}{$\cap$}{}
	\item Quick review of Trigonometric Identities
	\item Quick review of Differential Calculus
	\item Quick review of Integral Calculus
	\item Trignometry and Calculus relationship through Taylor Series and Maclaurin Series
\end{list}
	
\end{frame}

\begin{frame}
	\frametitle{ History of Mathematicians Involved}

	A number o\label{key}f mathematicians laid the building blocks for understanding the topics.  As you can see, it has taken painstaking work of many mathematicians for us to come to such a vast field. Below are are some of the mathematicians whose works contributed to these fields :
	\vspace{10pt}

	\begin{list}{$\int$}{}
		\item Jacob Bernoulli - Swiss mathematician invented $e$
		\item  Gottfried Wilhelm Leibniz  - Along with Newton invented Calculus
		\item Issac Newton - Along with Leibniz invented Calculus
		\item Carl Friedrich Gauss - One of the greatest mathematicians of all time - Imaginary numbers $\iu$
		\item Leonhard Euler - Worked on imaginary numbers
		\item Augustin-Louis Cauchy - Worked on imaginary numbers
		\item Greek, Chinese and Indian Mathematicians - Contributed to $\pi$
		\item James Gregory, Brook Taylor and Colin Maclaurin - Contributed to understanding of expansion series
	\end{list}
\end{frame}

\begin{frame}
	\frametitle{Trigonometry Brushup}
	Trigonometry is the study of three points.  There are three fundamental ratios that remain universally true across Trigonometry.  These are :
	\begin{list}{$\Upxi$}{}
		\item $\sin(x)$
		\item $\cos(x)$
		\item $\tan(x)$
	\end{list}

	We will quickly see in the graphs via Geogebra, how these things relate to each other.
	
\end{frame}

\begin{frame}
	\frametitle{Differential Calculus \& Limits Brushup}
	
Differential calculus is the study of rates. Rate of change of one variable with respect to another independent variable is the basis of differential calculus. Lets take a simple function $$y = f(x)= x^2$$

We want to study what happens to $f(x)$ as the value of x changes by a very small amount $\Delta x$
 $$\lim_{\Delta x \to 0} f(x+\Delta x) = (x+ \Delta x)^2 = x^2 +2.x.\Delta x +{\Delta x}^2$$
 
 So the change in $f(x)$ for a small change $\Delta x$ is be called as the differentiation of the function.  In other words, you are finding the slope of the function at any given point. so...
 
$$\frac{dy}{dx} = \lim_{\Delta x \to 0} \frac{f(x+\Delta x) -f(x)}{\Delta x} = 2x$$

\end{frame}

\begin{frame}
	\frametitle{Integral Calculus}
Integration is the opposite of derivatives. Integration calculates the area under a curve. In particular, the Indefinite Integral is the Accumulated Area Operator.  The area is
achieved by summing many rectangles of length $\Delta x = (b-a)/N$ and height
$f(a+i\Delta x )$.  Thus, the units of $$\int_a^b f(x) dx$$ is the units of $f(x)$ multiplied
by the units of $x$.  The integral introduces the peculiar-to-some idea of {\bf Negative Area}.  For example in integral calculus the area of a circle centered at the origin is NOT $$\pi r^2$$, it's ZERO as the bottom half of the circle is said to have negative area!	
\end{frame}

\begin{frame}
	\frametitle{Derivatives and Integrals for some functions} 
	\begin{center}
		\begin{tabular}{|c|c|}
			\hline
			Function & Derivative\\
			\hline
			$x^n$ & $nx^{n-1}$\\
			$e^x$ & $e^x$\\
			$ln (x)$ & $\frac{1}{x}$\\
			\hline
			$sin (x)$ & $cos (x) $\\
			$cos (x)$ & $-sin (x) $\\
			$-sin (x)$ & $-cos (x) $\\
			$-cos (x)$ & $sin (x) $\\
			\hline
		\end{tabular}
	\end{center}
	
	\begin{center}
		\begin{tabular}{|c|c|c|}
			\hline
			Function & Integration Form & Advanced Form\\
			\hline
			$x^n \;\; (n\neq -1)$ & $\frac{x^{n+1}}{n+1}$ & 
			$\int f^n(x) \frac{df}{dx} dx = \frac{f^{n+1}(x)}{n+1}$\\
			$x^{-1}$ & $ln |x| $ &
			$\int \frac{df/dx}{f} dx = ln |f(x)| $\\ 
			$e^x$ & $e^x$ & $\int e^{f(x)} \frac{df}{dx} dx = e^{f(x)} $\\
			$ln (x)$ & $x ln (x) - x $ &
			$\int ln(f(x))\frac{df}{dx} dx = f(x) ln (f(x)) - f(x) $\\ 
			\hline
			$sin(x)$ & $-cos(x)$ & $\int sin(f(x)) \frac{df}{dx} dx = cos(f(x))$\\
			$cos(x)$ & $sin(x)$ & \\
			\hline
		\end{tabular}
	\end{center}
\end{frame}

\begin{frame}
	\frametitle{The concept of exponential function $e^x$}
	
The number $e$ (sometimes called the natural number) is Euler's number. It is an important mathematical constant that is equal to approximately 2.718. When it is used as the base of a logarithm, the corresponding logarithm is called the natural logarithm, written as $\ln(x)$. \vspace{10pt}

Most of the definitions for $e$ involves calculus. One definition of $e$ will be seen later in this presentation. There are several important uses of $e$, one of which is the calculation of continuous compounding of interest in the field of finance. \vspace{10pt}

Jacob Bernoulli, the founder of the concept $e$ discovered this constant by asking many questions about the amount of money in a bank account after continous compounding of interest. He finally came up with the formula: \\

$$e^x = \lim_{n\to\infty}\Big(1+\dfrac{x}{n}\Big)^n$$
	
\end{frame}
\begin{frame}
	\frametitle{Taylor and Maclaurin Series \hspace{25pt} \textcolor{pink}{\Huge\(e^{\iu \pi} +1 = 0\)}}
	James Gregory and Brook Taylor invented the idea of infinite series expansion of functions.  Scottish mathematician Colin Maclaurin used the work extensively to study functions centered around zero.  The basic idea of Maclaurin series is that any function can be expanded into a infinite series centered around zero as given below :
	{\large {$$f(x) = f(0)+x.\dfrac{f'(0)}{1!}+ x^2.\dfrac{f''(0)}{2!} +x^3.\dfrac{f'''(0)}{3!} + \dots$$}}
	You can re-center it to any point $a$ on the x axis by shifting the equation to  :
	{\large {$$f(x) = f(a)+(x-a).\dfrac{f'(a)}{1!}+ (x-a)^2.\dfrac{f''(a)}{2!} +(x-a)^3.\dfrac{f'''(a)}{3!} + \dots$$}}
	
	But for this presentation we will stick to centering around zero. 
	\vspace{10pt}

	In the next three slides, we will see the application of the same for three basic functions: $e(x) \dots \sin(x) \dots \cos(x)$
\end{frame}

\begin{frame}
	\frametitle{Taylor Series for e(x) \hspace{25pt} \textcolor{pink}{\Huge\(e^{\iu \pi} +1 = 0\)}}
	\begin{center}
		\begin{tabular}{|c|c|c|c|c|c|}
			\hline
			Order &$f(x)$ & $f'(x)$ & $f''(x)$ & $f'''(x)$ & $f''''(x)$ \\
			\hline
			Function & $e^x$  & $e^x$   & $e^x$    & $e^x$     & $e^x$      \\
			\hline
			Value at 0 & 1      & 1       & 1        & 1         & 1          \\
			\hline
		\end{tabular}
	\end{center}
	\vspace{20pt}
	Applying the Maclaurin formula we see that the infinite series expansion for $e^x$ is :

	{\large $$f(x) = f(0)+x.\dfrac{f'(0)}{1!}+ x^2.\dfrac{f''(0)}{2!} +x^3.\dfrac{f'''(0)}{3!} + \dots$$
	\large $$e^x = 1+x.\dfrac{1}{1!}+ x^2.\dfrac{1}{2!} +x^3.\dfrac{1}{3!} + \dots$$}

	By extension:
	{\large $$e^{{\iu}.x} = 1+\iu.x.\dfrac{1}{1!}- x^2.\dfrac{1}{2!} -\iu.x^3.\dfrac{1}{3!} + x^4.\dfrac{1}{4!} \dots$$}
\end{frame}

\begin{frame}
	\frametitle{Taylor Series for $\sin(x)$ \hspace{25pt} \textcolor{pink}{\Huge\(e^{\iu \pi} +1 = 0\)}}
	\begin{center}
		\begin{tabular}{|c|c|c|c|c|c|}
			\hline
			Order & \(f(x)\)    & \(f'(x)\)   & \(f''(x)\)   & \(f'''(x)\)  & \(f''''(x)\) \\
			\hline
			Function & \(\sin(x)\) & \(\cos(x)\) & \(-\sin(x)\) & \(-\cos(x)\) & \(\sin(x)\)  \\
			\hline
			Value at 0 & 0           & 1           & 0            & -1           & 0            \\
			\hline
		\end{tabular}
	\end{center}
	\vspace{20pt}
	Applying the Maclaurin formula we see that the infinite series expansion for $\sin(x)$ is :
	{\large $$f(x) = f(0)+x.\dfrac{f'(0)}{1!}+ x^2.\dfrac{f''(0)}{2!} +x^3.\dfrac{f'''(0)}{3!} + \dots$$
	\large $$\sin(x) = 0+x.\dfrac{1}{1!} -x^3.\dfrac{1}{3!} + x^5.\dfrac{1}{5!} +\dots$$}	
	By extension: 
	{\large $$\iu.\sin(x) = 0+\iu.x.\dfrac{1}{1!} -\iu.x^3.\dfrac{1}{3!} + \iu.x^5.\dfrac{1}{5!} +\dots$$}

\end{frame}

\begin{frame}
	\frametitle{Taylor Series for $\cos(x)$ \hspace{25pt} \textcolor{pink}{\Huge\(e^{\iu \pi} +1 = 0\)}}
	\begin{center}
		\begin{tabular}{|c|c|c|c|c|c|}
			\hline
			Order & $f(x)$    & $f'(x)$    & $f''(x)$  & $f'''(x)$ & $f''''(x)$ \\
			\hline
			Function & $\cos(x)$ & $-\sin(x)$ & $-cos(x)$ & $\sin(x)$ & $\cos(x)$  \\
			\hline
			Value at 0 & 1         & 0          & -1        & 0         & 1          \\
			\hline
		\end{tabular}
	\end{center}
	\vspace{20pt}
	Applying the Maclaurin formula we see that the infinite series expansion for $\cos(x)$ is :

	\large $$f(x) = f(0)+x.\dfrac{f'(0)}{1!}+ x^2.\dfrac{f''(0)}{2!} +x^3.\dfrac{f'''(0)}{3!} + \dots$$

	\large $$\cos(x) = 1- x^2.\dfrac{1}{2!} +x^4.\dfrac{1}{4!} + \dots$$

\end{frame}

\begin{frame}
	\frametitle{Bringing it all together \hspace{25pt} \textcolor{pink}{\Huge\(e^{\iu \pi} +1 = 0\)}}
	\begin{center}
	$\cos(x) + \iu.\sin(x) = 1+\iu.x.\dfrac{1}{1!}- x^2.\dfrac{1}{2!} -\iu.x^3.\dfrac{1}{3!} +x^4.\dfrac{1}{4!} + \iu.x^5.\dfrac{1}{5!} +\dots = e^{\iu.x}$
	\end{center}

	From the equations on the previous slides, we can see that \dots
	\begin{center}
		$\cos(x) + \iu.\sin(x) = e^{\iu.x}$ 
	\end{center} 
	If $x = \pi$ \dots
	\begin{center}
		$\cos(\pi) + \iu.\sin(\pi) = e^{\iu.\pi}$ \\
	\end{center}
	Then \dots
	\begin{center}
		$e^{\iu.\pi} + 1 = 0$
	\end{center}
\end{frame}




\begin{frame}
	\frametitle{Quick review of Complex Numbers \hspace{25pt} \textcolor{pink}{\Huge\(e^{\iu \pi} +1 = 0\)}}
	A complex number as a real part and an imaginary part.  Lets assume two complex numbers :
	\begin{list}{$\odot$}{}
		\item  $z = a+\iu b$
		\item  $w = c+\iu d$
	\end{list}
Then the following rules apply to complex numbers
	\begin{list}{$\odot$}{}
		\item $ z+w = (a+c) + \iu(b+d) $\\
		
		\item $ z-w = (a-c) + \iu(b-d) $ \\
		
		\item $ z*w = (ac-bd) +\iu(bc+ad)$
		
		\item $z^2 = (a^2-b^2) +\iu(2ab)$ \\
		
		\item magnitude of $|z| = r = \sqrt{a^2 +b^2}$ 
		
		\item The trignometric representation is $z = r(cos(\theta)+\iu\sin(\theta))$ where $\theta = tan^{-1}(\frac{b}{a})$
		
	\end{list}
\end{frame}

\begin{frame}
	\frametitle{A Graphical Proof \hspace{25pt} \textcolor{pink}{\Huge\(e^{\iu \pi} +1  = 0\)}}
	%\begin{center}
	Now I will show you all a graphical representation of this beautiful mathematical formula on Geogebra. The graphical proof of the equation is visually pleasing. We will now switch to Geogebra to go over the second way of proving the equation.
	%\end{center}
\end{frame}


\begin{frame}
	\begin{center}
		\Huge \textcolor{blue}{Thank You} \\
		\vspace{25pt}
		\huge \textcolor{pink}{Don't be a : ${\dfrac{d^3x}{dt^3}}$}
	\end{center}
\end{frame}

\end{document}
